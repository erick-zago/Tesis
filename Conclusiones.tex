\chapter{Conclusiones}
\label{cap:conclusiones}

Las bases en desarrollo de software, bases de datos, estructuras de datos,
lenguajes de programación y, por encima de todo, análisis y solución de
problemas que obtuve dentro de la carrera de Ciencias de la Computación, me
ayudaron a que el proyecto que presento en este trabajo se realizara de manera
exitosa.

La implementación de este tipo de proyectos de reestructuración de información 
está siendo un tema recurrente en muchas instituciones ya que no se tiene, en muchos casos, 
una cultura de centralización de información y sobre todo de gobierno de datos. 
Las instituciones deben de poner especial atención a la consistencia de la información de 
manera que todas las areas tengan acceso a los mismos datos, evitando así que cada área 
opere de manera indepeniente y tenga sus propios procesos o que presenten diferentes reportes
por tener acceso a diferentes fuentes de información.

Como se mencionó anteriormente, los conocimientos adquiridos durante la carrera de ciencias de 
la computación me permitieron llevar a cabo con éxito la implementación de este proyecto. La 
metodología aprendida en la ingeniería de software y las clases de bases de datos permitieron realizar 
un análisis detallado de la problematica y de esta manera proponer un modelo nuevo para la institución
financiera. 

Agradezco todas y cada una de las clases porque gracias a ellas tuve el éxito que se requería para 
el proyecto. 

\cleardoublepage

%%% Local Variables:
%%% TeX-master: "Tesis"
%%% End:
