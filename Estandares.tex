\chapter{Estándares de programación}
\label{cap:estandares}

La convención de nombres tiene la ventaja de mejorar la lectura y comprensión de
la aplicación, tanto para el desarrollo como para el mantenimiento; así mismo,
tiene soporte para ciertos aspectos de la operación debido a que permite una
mejor interpretación de la información contenida en cada elemento de la base de
datos o procesos de ETL. Varios aspectos son posibles o más fáciles si se tienen
definidos los estándares y convenciones de nombres:

\begin{itemize}
\item Actualizaciones (cambio de versiones) a la base de datos o software.
\item Cambio al modelo de datos físico o lógico.
\item Mover una aplicación de un servicio a otro.
\item Mejor administración de la base de datos (uso de utilerías y
  \emph{scripts}).
\end{itemize}

Los aspectos que serán cubiertos en este capitulo serán:

\begin{itemize}
\item Lineamientos de la base de datos.
\item Uso de nombres y convenciones.
\item Reglas para crear nombres de los objetos de las bases de datos.
\item Recomendaciones para crear nombres de proyectos.
\end{itemize}

\section{Lineamientos de base de datos}

Algunos de los lineamientos que se siguieron al momento de realizar la
construcción de la base de datos se enlistan a continuación:

\begin{enumerate}

\item En cada construcción de consultas, se verificó siempre los planes de
  ejecución; es decir, reducir al máximo el uso de memoria y capacidad de las
  máquinas. Además evitar que las consultas repondieran con escaneo de índices
  (\texttt{index scan}) o escaneo de tablas (\texttt{table scan}).

\item Evitar el uso del enunciado \texttt{WHERE ... IN} y utilizar \texttt{WHERE
    EXIST}.

\item Evitar el uso del \texttt{OR} y sustituirlo por otro tipo de consulta o
  por el enunciado \texttt{CASE}.

\item Evitar hacer uso de los operadores como \texttt{NOT} o
  \texttt{\textless\textgreater}.

\item En la ejecución de consultas, después del enunciado \texttt{WHERE}, hacer
  siempre referencia a las columnas con índices.

\item En enunciados de inserción de registros (\texttt{INSERT}) siempre colocar
  el nombre de las columnas. Lo anterior ayuda a evitar problemas de
  carga/inserción derivado de cambios en la estructura de la tabla.

\item Evitar el uso de comodines; es decir, escribir siempre el nombre de las
  columnas requeridas para reducir el tráfico en la red.

\item Mantener los enunciados SQL lo más concisos posibles.

\item Para el proceso de carga masiva de datos, se recomendó borrar primero los
  índices antes de iniciar la carga, reconstruyéndolos al final del proceeso.

\item Siempre comentar el código generado.

\item No usar el prefijo ``\texttt{\_sp}'' para nombrar procedimientos
  alamacenados (\texttt{stored procedures}).

\item Uso de limitadores en la obtención de registros de cualquier tabla.

\item Evitar el uso de tablas temporales en el procesamiento de datos. En
  algunos casos se utilizaron tablas temporales en lugar de consultas complejas.

\item Evitar el uso de cursores. En su lugar se utilizaron enunciados
  \texttt{WHILE}.

\item Después de la cláusula \texttt{ORDER BY}, poner el nombre de las columnas
  en lugar de valores numericos que hagan referencia al orden de las columnas.

\item Cumplir con los estándares definidos para la base de datos.

\end{enumerate}

\section{Reglas para crear nombres de objetos en la base de datos}

Las siguientes reglas fueron definidas para realizar objetos dentro de la base
de datos.

\begin{itemize}

\item \textbf{Caracteres.} Se debía cumplir que:

  \begin{itemize}
  \item Los nombres de los elementos debían consistir solamente de letras(A-Z),
    números (0-9) y guión bajo (\_).
  \item Los nombres no debían tener espacios.
  \item Los nombres no debían tener comas o comillas
  \end{itemize}

\item \textbf{Longitud.} Si bien la longitud de los nombres es propia de cada
  herramienta de software, en el caso de Oracle no puede ser mayor a 30 bytes
  con las siguientes excepciones: nombre de base de datos hasta 8 bytes y nombre
  de ligas de base de datos hasta 128 bytes.

\item \textbf{Palabras reservadas.} No estaba permitido el uso de palabras
  reservadas.

\item \textbf{Separadores.} Uso del caracter `\texttt{\_}' para nombres de
  objetos de base de dato que estuvieran compuestos de varias palabras. No usar
  `\texttt{\_}' en el nombre de la base de datos.

\item \textbf{Identificadores numericos (\#\#).} En las convenciones de números,
  el símbolo \texttt{\#\#} es usado muchas veces para indicar que varios objetos
  de la base de datos deben de diferenciarse unos de otros.

\end{itemize}

\section{Recomendaciones para crear nombres de proyectos}

Una de las metas de utilizar convenciones de nombres es hacer que los nombres
utilizados sean lo más simple posible, ya que el nombre debe ser fácil de
recordar y debe seguir ciertas reglas que sean igualmente fáciles de explicar.

Las siguientes reglas no fueron obligatorias para el proyecto, pero se tomaron
algunas de sus recomendaciones.

\begin{itemize}

\item \textbf{Nombres de proyectos.} Para la definición de nombres se deben
  cumplir dos metas principales: el nombre del objeto debe ser tan corto como
  sea posible y debe ser autodescriptivo. Una regla que hay que seguir es que
  cuando el nombre no pueda ser corto se debe elegir un nombre autodescriptivo
  pensando en que probablemente no sea la misma persona la que realice la
  implementación y el mantenimiento.

\item \textbf{Verbos.} Si los nombres de los objetos contienen un verbo, siempre
  se tiene que iniciar la palabra con él. Por ejemplo es mejor decir
  \texttt{Buscar\_Nombre\_Cliente} que \texttt{Nombre\_Cliente\_Buscar}.

\item \textbf{Palabras simples.} Utilizar palabras simples para describir un
  objeto. Usar \texttt{Buscar\_Complemento} o sólo \texttt{Complemento} en lugar
  de \texttt{Buscar\_Y\-\_Complementar}.

\item \textbf{Expresiones cotidianas.} Las expresiones cotidianas no deben ser
  usadas ya que pueden causar confusión al momento de leer o implementar.

\item \textbf{Uso de formas singulares y plurales.} Las formas singulares son
  más comunes que las plurales en el uso diario; son mas cortas y simples de
  entender, por lo que es recomendable usarlas en lugar de las plurales. Como
  una regla de consistencia, cuando se define el nombre de la base de datos se
  puede utilizar la forma singular o plurar, pero no ambas.

\item \textbf{Uso de abreviaciones.} Si es necesario tener que usar
  abreviaciones, las siguientes recomendaciones nos pueden ayudar a definir una
  buena abreviación (tomaremos como ejemplo la palabra \texttt{ERROR}).

  \begin{itemize}

  \item Obtener todas las consonantes o caracteres especiales de la palabra
    (\texttt{ERROR} sería \texttt{RRR}).

  \item Si la palabra original comienza con vocal, mantener la vocal en su lugar
    (\texttt{ERROR} sería \texttt{ERRR}).

  \item Si la regla antes descrita resulta en una abreviación que contiene más
    de dos consonantes idénticas, solo dos consonantes deben permanecer
    (\texttt{ERROR} sería \texttt{ERR}).

  \item Si el resultado de la abreviación es díficil de entender, usar una
    abreviación que sea más comprensible. Por ejemplo usar \texttt{PROD} para
    describir producto en lugar de \texttt{PRDCT}.

  \end{itemize}

\end{itemize}

\cleardoublepage

%%% Local Variables:
%%% TeX-master: "Tesis"
%%% End:
