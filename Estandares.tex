\chapter{Estándares de programación}
\label{cap:estandares}

La convención de nombres tiene la ventaja de mejorar la lectura y comprensión de
la aplicación, tanto para el desarrollo como para el mantenimiento; así mismo,
tiene soporte para ciertos aspectos de la operación debido a que permite una
mejor interpretación de la información contenida en cada elemento de la base de
datos o procesos de ETL. Varios aspectos son posibles o más fáciles si se tienen
definidos los estándares y convenciones de nombres:

\begin{itemize}
\item Actualizaciones (cambio de versiones) a la base de datos o software.
\item Cambio al modelo de datos físico o lógico.
\item Mover una aplicación de un servicio a otro.
\item Mejor administración de la base de datos (uso de utilerías y
  \emph{scripts}).
\end{itemize}

Los aspectos que serán cubiertos en este capitulo serán:

\begin{itemize}
\item Lineamientos de la base de datos.
\item Uso de nombres y convenciones.
\item Reglas para crear nombres de los objetos de las bases de datos.
\item Recomendaciones para crear nombres de proyectos.
\end{itemize}

\section{Lineamientos de base de datos}

Algunos de los lineamientos que se siguieron al momento de realizar la
construcción de la base de datos se enlistan a continuación:

\begin{enumerate}

\item En cada construcción de consultas, se verificó siempre los planes de
  ejecución; es decir, reducir al máximo el uso de memoria y capacidad de las
  máquinas. Además evitar que las consultas respondieran con escaneo de índices
  (\texttt{index scan}) o escaneo de tablas (\texttt{table scan}).

\item Evitar el uso del enunciado \texttt{WHERE \ldots{} IN} y utilizar
  \texttt{WHERE EXIST}.

\item Evitar el uso del \texttt{OR} y sustituirlo por otro tipo de consulta o
  por el enunciado \texttt{CASE}.

\item Evitar hacer uso de los operadores como \texttt{NOT} o
  \texttt{\textless\textgreater}.

\item En la ejecución de consultas, después del enunciado \texttt{WHERE}, hacer
  siempre referencia a las columnas con índices.

\item En enunciados de inserción de registros (\texttt{INSERT}) siempre colocar
  el nombre de las columnas. Lo anterior ayuda a evitar problemas de
  carga/inserción derivado de cambios en la estructura de la tabla.

\item Evitar el uso de comodines; es decir, escribir siempre el nombre de las
  columnas requeridas para reducir el tráfico en la red.

\item Mantener los enunciados SQL lo más concisos posibles.

\item Para el proceso de carga masiva de datos, se recomendó borrar primero los
  índices antes de iniciar la carga, reconstruyéndolos al final del proceso.

\item Siempre comentar el código generado.

\item No usar el prefijo ``\texttt{\_sp}'' para nombrar procedimientos
  alamacenados (\texttt{stored procedures}).

\item Uso de limitadores en la obtención de registros de cualquier tabla.

\item Evitar el uso de tablas temporales en el procesamiento de datos. En
  algunos casos se utilizaron tablas temporales en lugar de consultas complejas.

\item Evitar el uso de cursores. En su lugar se utilizaron enunciados
  \texttt{WHILE}.

\item Después de la cláusula \texttt{ORDER BY}, poner el nombre de las columnas
  en lugar de valores numéricos que hagan referencia al orden de las columnas.

\item Cumplir con los estándares definidos para la base de datos.

\end{enumerate}

\section{Reglas para crear nombres de objetos en la base de datos}

Las siguientes reglas fueron definidas para realizar objetos dentro de la base
de datos.

\begin{itemize}

\item \textbf{Caracteres.} Se debía cumplir que:

  \begin{itemize}
  \item Los nombres de los elementos debían consistir solamente de letras(A-Z),
    números (0-9) y guión bajo (\_).
  \item Los nombres no debían tener espacios.
  \item Los nombres no debían tener comas o comillas
  \end{itemize}

\item \textbf{Longitud.} Si bien la longitud de los nombres es propia de cada
  herramienta de software, en el caso de Oracle no puede ser mayor a 30 bytes
  con las siguientes excepciones: nombre de base de datos hasta 8 bytes y nombre
  de ligas de base de datos hasta 128 bytes.

\item \textbf{Palabras reservadas.} No estaba permitido el uso de palabras
  reservadas.

\item \textbf{Separadores.} Uso del carácter `\texttt{\_}' para nombres de
  objetos de base de dato que estuvieran compuestos de varias palabras. No usar
  `\texttt{\_}' en el nombre de la base de datos.

\item \textbf{Identificadores numéricos (\#\#).} En las convenciones de números,
  el símbolo \texttt{\#\#} es usado muchas veces para indicar que varios objetos
  de la base de datos deben de diferenciarse unos de otros.

\end{itemize}

\section{Recomendaciones para crear nombres de proyectos}

Una de las metas de utilizar convenciones de nombres es hacer que los nombres
utilizados sean lo más simple posible, ya que el nombre debe ser fácil de
recordar y debe seguir ciertas reglas que sean igualmente fáciles de explicar.

Las siguientes reglas no fueron obligatorias para el proyecto, pero se tomaron
algunas de sus recomendaciones.

\begin{itemize}

\item \textbf{Nombres de proyectos.} Para la definición de nombres se deben
  cumplir dos metas principales: el nombre del objeto debe ser tan corto como
  sea posible y debe ser autodescriptivo. Una regla que hay que seguir es que
  cuando el nombre no pueda ser corto se debe elegir un nombre autodescriptivo
  pensando en que probablemente no sea la misma persona la que realice la
  implementación y el mantenimiento.

\item \textbf{Verbos.} Si los nombres de los objetos contienen un verbo, siempre
  se tiene que iniciar la palabra con él. Por ejemplo es mejor decir
  \texttt{Buscar\_Nombre\_Cliente} que \texttt{Nombre\_Cliente\_Buscar}.

\item \textbf{Palabras simples.} Utilizar palabras simples para describir un
  objeto. Usar \texttt{Buscar\_Complemento} o sólo \texttt{Complemento} en lugar
  de \texttt{Buscar\_Y\-\_Complementar}.

\item \textbf{Expresiones cotidianas.} Las expresiones cotidianas no deben ser
  usadas ya que pueden causar confusión al momento de leer o implementar.

\item \textbf{Uso de formas singulares y plurales.} Las formas singulares son
  más comunes que las plurales en el uso diario; son mas cortas y simples de
  entender, por lo que es recomendable usarlas en lugar de las plurales. Como
  una regla de consistencia, cuando se define el nombre de la base de datos se
  puede utilizar la forma singular o plural, pero no ambas.

\item \textbf{Uso de abreviaciones.} Si es necesario tener que usar
  abreviaciones, las siguientes recomendaciones nos pueden ayudar a definir una
  buena abreviación (tomaremos como ejemplo la palabra \texttt{ERROR}).

  \begin{itemize}

  \item Obtener todas las consonantes o caracteres especiales de la palabra
    (\texttt{ERROR} sería \texttt{RRR}).

  \item Si la palabra original comienza con vocal, mantener la vocal en su lugar
    (\texttt{ERROR} sería \texttt{ERRR}).

  \item Si la regla antes descrita resulta en una abreviación que contiene más
    de dos consonantes idénticas, solo dos consonantes deben permanecer
    (\texttt{ERROR} sería \texttt{ERR}).

  \item Si el resultado de la abreviación es difícil de entender, usar una
    abreviación que sea más comprensible. Por ejemplo usar \texttt{PROD} para
    describir producto en lugar de \texttt{PRDCT}.

  \end{itemize}

\end{itemize}

\section{Nombres estándar para la base de datos}

Parte importante dentro de la programación son los nombres usados en la base de
datos; a continuación se definen algunas recomendaciones que se deben seguir
para tener la información en nuestro sistema de una forma entendible y fácil de
leer.

\subsection{Estándares para la creación de un modelo lógico}

\subsubsection{Atributos}

\begin{enumerate}

\item Todos los nombres lógicos deben ser expresados en términos de negocio con
  las siguientes excepciones:

  \begin{itemize}
  \item Acrónimos de negocio similares pueden ser usados de manera limitada.
  \item El nombre \texttt{ID} debe ser usado siempre para identificadores.
  \end{itemize}

\item Todos los nombres deben de iniciar con letra mayúscula.

\item Todos los nombres son singulares.

\item Las entidades deben de ser únicas dentro del modelo.

\item No se deben usar abreviaciones al menos que sea requerido por
  consideraciones de espacio y se deberán seguir las recomendaciones de la
  sección anterior.

\item Los nombres deben de consistir e iniciar con caracteres alfabéticos. Los
  números no son recomendables pero pueden ser usados después del inicio en caso
  de ser requeridos. Caracteres especiales no deben de ser usados.

\item Múltiples términos en el nombre deben estar separados por espacio, las
  diagonales o guiones no pueden ser usados.

\end{enumerate}

\subsubsection{Entidades}

\begin{enumerate}

\item Un nombre de entidad debe ser un nombre o una frase en forma singular.

\item Todos los nombres de entidades son nombres de negocio y pueden terminar en
  una palabra que los clasifique.

\item Entidades asociadas deben ser creadas usando un nombre con un significado
  de negocio en lugar de la concatenación de 2 entidades.

\item La entidad asociada debe ser sufijo con una abreviación para la referencia
  cruzada.

\end{enumerate}

\subsubsection{Relaciones}

Todas las relaciones tienen significado de una frase verbal y deben estar en
minúsculas.

\subsubsection{Definiciones}

Las definiciones deben ser una frase detallada de los atributos que típicamente
contiene referencias al sujeto componente del atributo. La definición debe
consistir de varios enunciados que describan concisamente los datos. Ejemplos de
datos deben proporcionarse siempre que sea posible.

\subsubsection{Propiedades definidas por el usuario}

Para todos los objetos, los comentarios deben ser proporcionados en las
propiedades definidas por el usuario. Las funciones definidas por el usuario
(\texttt{UDF}) son muy similares a los procedimientos almacenados con la
diferencia de que estos pueden ser usados en los enunciados \texttt{SELECT}; si
no fuera así, tanto los procedimientos almacenados como las funciones definidas
por el usuario serían iguales. Las funciones definidas por el usuario deberán de
tener el sufijo \texttt{\_udf}.

\subsection{Estándares para la creación de un modelo físico}

\subsubsection{Nombre de tablas}

La regla principal para crear los nombres de tablas físicas es que deben estar
basadas en los nombres de las tablas descritas en los estándares del modelo
lógico visto en la sección anterior, con excepción de las siguientes reglas.

\begin{itemize}

\item El guión bajo reemplazará a los espacios que dividen los términos en el
  nombre.

\item Uso de términos estándar apropiados. Los nombres deben ser amigables al
  usuario y consistentes con el mismo término utilizado por el modelo lógico.

\item Los nombres no deben de contener palabras reservadas propias de la base de
  datos como: \texttt{type}, \texttt{name}, \texttt{column}, \texttt{row}, etc.

\item No se deben usar acentos, diéresis ni la letra \texttt{ñ}.

\end{itemize}

\subsubsection{Nombres de elementos de las tablas}

Los fundamentos para crear los nombres de los elementos físicos de la base de
datos deben de estar basados en los nombres de los atributos de negocio
definidos en los estándares para el modelo lógico con las siguientes
excepciones:

\begin{itemize}

\item El carácter `\texttt{\_}' reemplazará a los espacios que separan los
  términos en el nombre

\item Uso de términos estándar apropiados. Los nombres deben de ser amigables al
  usuario y consistentes con el mismo término utilizado en la definición del
  modelo lógico.

\item Las llaves técnicas o sustitutas deben tener el sufijo \texttt{\_TK}.

\item Los nombres de las restricciones (\emph{constrains}) deben ser de la
  siguiente forma:

  \begin{itemize}

  \item \texttt{Nombretabla} es el nombre de la tabla para el cual la
    restricción aplica.

  \item \texttt{tipo} identifica el tipo de restricción y solamente puede ser
    una de las siguientes opciones:

    \begin{itemize}
    \item \texttt{pk}, llave primaria.
    \item \texttt{fk}, llave foránea.
    \item \texttt{c}, validar restricción.
    \end{itemize}

  \item \texttt{desc} es información adicional correspondiente al nombre de las
    columnas usadas en la restricción o a su uso.

  \end{itemize}

\end{itemize}

\subsubsection{Restricciones por omisión y de comprobación}

Se debe de usar el nombre de la columna a la cuál la restricción está unida y
colocar un sufijo \texttt{def} para restricciones por omisión (\emph{default}) y
\texttt{chk} para restricciones de comprobación (\emph{check}) respectivamente.

\subsubsection{Nombres de procedimientos almacenados}

Los procedimientos almacenados (\emph{stored procedures}) realizan tareas por el
usuario por lo que es bueno seguir una serie de reglas para la definición de los
nombres. Las reglas que se deberán seguir son las siguientes:

\begin{itemize}

\item Usar la plantilla \texttt{sp\_x\_yyyyyy} para nombrar a los procedimientos
  almacenados, donde:

  \begin{itemize}

  \item \texttt{sp} denota que se trata de un procedimiento almacenado.

  \item \texttt{x} denota el tipo de procedimiento almacenado a ejecutar:

    \begin{itemize}
    \item \texttt{d} procedimiento de borrado.
    \item \texttt{i} procedimiento de inserción.
    \item \texttt{s} procedimiento de selección.
    \item \texttt{u} procedimiento de actualización.
    \item \texttt{x} procedimiento que combina inserción, selección y
      actualización.
    \end{itemize}

  \item \texttt{yyyyyy} descripción del procedimiento almacenado.

  \item Usar un verbo para describir la acción del procedimiento almacenado.

  \item Opcionalmente se puede usar un prefijo que agrupe los procedimientos
    almacenados de una misma tabla.

  \end{itemize}

\end{itemize}

\subsubsection{Variables}

Para variables que almacenan el contenido de las columnas, se puede utilizar la
misma regla que utilizamos para nombrar columnas. A continuación se describen
las reglas generales:

\begin{itemize}

\item No definir nombres largos y complicados para las tablas u otros objetos de
  las bases de datos.

\item No usar espacios en los nombres de los objetos de la base de datos ya que
  los espacios hacen confuso el acceso a las herramientas y aplicaciones de la
  base de datos. En el caso obligatorio de tener que usar espacios, asegurarse
  de colocar el nombre dentro de corchetes.

\item No usar palabras reservadas para nombrar objetos de base de datos, porque
  esto puede resultar en situaciones impredecibles.

\end{itemize}

\cleardoublepage

%%% Local Variables:
%%% TeX-master: "Tesis"
%%% End:
