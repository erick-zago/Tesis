\chapter*{Marco Teórico}
\label{cap:mteorico}

El marco teórico del proyecto se basa en laarquitecura de datos, que en
tecnología de la información se compone de los modelos, políticas, reglas y/o
estándares que determinan qué datos se recolectan y cómo se guardan, ordenan,
integran y usan en los sistemas de datos de una institución.

Hablar de datos es entrar en un mundo complejo donde existen diferentes
perspectivas de datos, dependiendo del uso que se les quiera dar o de las
personas que lo utilizan. Cada grupo de personas tiene su propia perspectiva
sobre el manejo de datos: manejo de grandes volúmenes, acceso al detalle de los
datos de manera instantánea, manejo de la integridad, datos de acceso exclusivo,
etc. La arquitectura de datos nos ayuda a que estos diferentes tipos de datos y
diferentes necesidades puedan coexistir de una forma conjunta de acuerdo a las
necesidades de cada área o empresa.

Actualmente no existe ningún secreto para el manejo de datos y su respectiva
arquitectura; en ambos casos, es importante entender los datos en términos de su
infraestructura; es decir, se requiere conocer la infraestructura que rodea a
los datos para llevar a cabo un uso adecuado de los mismos.

Así mismo, es importante saber que dentro de cualquier empresa u organización se
pueden encontrar diferentes tipos de datos: estructurados y no
estructurados. Los primeros son datos predecibles y que normalmente son
manejados en una base de datos (SMBD): registros, atributos, llaves, índices,
etc. Por su parte, los datos no estructurados no son predecibles y, como su
nombre lo indica, no tienen una estructura bien definida, usualmente son de
dificil acceso y generalmente se requiere una búsqueda más profunda para hacer
consultas; por ejemplo, una cadena de carácteres en un texto libre.

Todos estos conceptos serán detallados y aplicados en los capítulos del proyecto
presentado.

\cleardoublepage

%%% Local Variables:
%%% TeX-master: "Tesis"
%%% End:
